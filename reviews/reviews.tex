\documentclass[8pt]{article}
\usepackage{lineno} 
\usepackage{enumitem}
\usepackage{indentfirst}
\usepackage{geometry}
 \geometry{
 a4paper,
 total={210mm,297mm},
 left=20mm,
 right=20mm,
 top=20mm,
 bottom=10mm,
 }
 
\usepackage{titlesec}
% \titlespacing{command}{left spacing}{before spacing}{after spacing}[right]
%  spacing: how to read {12pt plus 4pt minus 2pt}
%           12pt is what we would like the spacing to be
%           plus 4pt means that TeX can stretch it by at most 4pt
%           minus 2pt means that TeX can shrink it by at most 2pt
%           This is one example of the concept of, 'glue', in TeX
%           reference: [http://tex.stackexchange.com/questions/53338/reducing-spacing-after-headings]
\titlespacing\section{0pt}{5pt plus 4pt minus 2pt}{5pt plus 2pt minus 2pt}
\titlespacing\subsection{0pt}{5pt plus 4pt minus 2pt}{5pt plus 2pt minus 2pt}


% Preamble
\title{Submission's Reviews for the ISWC 2015}
\author{Miguel Angel Perez-Xochicale}
\date{28 May 2015}

% Content
\begin{document}
\maketitle
\linenumbers


\section{mail}

Dear Miguel Angel Perez-Xochicale. 

We are sorry to inform you that your submission 
208: ``Dexterity Assesment for Salsa Dancers Through the Time-delay Embedded Phase Space Representation''
has not been accepted for publication at ISWC 2015 Papers, Note and Brief.

121 papers were submitted to ISWC 2015 Papers, Note and Brief, of which 14 (8 conditionally with shepherding) 
have been accepted as long papers, 16 as short papers (13 conditionally with shepherding), and 
9 as brief (7 conditionally with shepherding). The preliminary acceptance rate for oral presentation is 25\%.

We hope these will be helpful in revising your submission. We would like to thank you for submitting your work, 
and in spite of the negative outcome on this occasion, we do hope that you will continue to keep ISWC in mind 
when reporting your work in the future.

Kristof Van Laerhoven and Tsutomu Terada
ISWC 2015 PC Co-Chairs

\subsection{Feedback from Program Committee Meeting}

   Dear Autors of ``Dexterity Assesment for Salsa Dancers Through the Time-delay Embedded 
Phase Space Representation'',
   Thank you for your submission to ISWC 2015. Although your submission
   contains an interesting application, the review committee critically
   discussed your paper and decided to reject it. The reasons are detailed
   in the individual reviewers' reports. Globally, the following points were
   the most relevant for the final decision:
   
   \begin{itemize}[topsep=0pt,itemsep=-1ex,partopsep=1ex,parsep=1ex]
    \item The study has important d (number of subjects low, evaluation limited),
   the method is not well motivated and is not directly transferable to
   other applications. 
    \item The innovation is small compared to other contributions that have been
   submitted to ISWC.
    \item As a minor point, the writing is hard to comprehend sometimes and needs
   to be tidied up.
   \end{itemize}

We would recommend to work on the above mentioned deficiencies and to
then resubmit your paper to another conference.


%%%%%%%%%%%%%%%%%%%%%%%%%%%%%%%%%%%%%%%
\section{Review 1}
\subsection{Confidence}
3  (Very confident - I am knowledgeable in the area)
   
\subsection{Contribution to ISWC}
\textbf{Contribution to ISWC} \\
   The paper presents a methodology for dexterity assessment of dancers
   using inertial measurement data and dynamical system analysis. The paper
   presents a very interesting application, but for a limited audience. In
   terms of "raising the bar" regarding the paper's method for Human
   Activity Recognition, the contribution is not sufficiently high to make a
   substantial contribution there.

\subsection{Overall Rating}
\textbf{Overall Rating} \\
2  (Probably reject: I would argue for rejecting this paper.)

\subsection{The Review}
\textbf{The Review}
\begin{itemize}[topsep=0pt,itemsep=-1ex,partopsep=1ex,parsep=1ex]
    \item Very interesting application
    \item Small audience interested in the main application
    \item Method not directly transferable to other application domains, this
   should be discussed
    \item Contribution also does not sufficiently "raise the bar" in activity
   recognition
    \item Evaluation is rather limited - it is hard to argue differences in skill
   level based on data from just one person in each group, there has to be a
   substantial extension of the data collection
   \end{itemize}


   
%%%%%%%%%%%%%%%%%%%%%%%%%%%%%%%%%%%%%%%   
\section{Review 2}
\subsection{Confidence}
3  (Very confident - I am knowledgeable in the area)

\subsection{Contribution to ISWC}
   The paper's aim is to recognize the experience level of salsa dansers -
   the aproach proposed is based on "time-delay-embedded-phase-space"
   
\subsection{Overall Rating}
   2  (Probably reject: I would argue for rejecting this paper.)
   
\subsection{The Review}
   After reading the paper it is not quite clear to me what the major
   scientific contribution of the paper is supposed to me. In the following
   I list the issues I have

   \begin{itemize}[topsep=0pt,itemsep=-1ex,partopsep=1ex,parsep=1ex]
   \item overal the paper is not well written. In fact is rather verbose and
   given the content the paper should have been at most a 4-page submission
   \item the term ``dexterity'' seems not what is done - at least not from my
   understanding of the english word - I would suggest a different word like
   ``level of expertise''.
   \item the experiments are anecdotal at best - it starts with the fact that
   for two of the three ``expertise levels'' there is one person each making
   the entire experimental results questionable. Additionally only
   qualitative results are shown making it entirely unclear if the approach
   is sensible or not.
   \item additionally, from the writing it is not clear to me if there is any
   novelty in the approach using the time-delay embedding matrix - maybe
   there is novelty - it is not made clear what the authors claim here.
   \end{itemize}
   
   So the only conclusion I can draw at this point that this preliminary
   version of the work (and writing) cannot be accepted as is. 
   Also - unfortunately - the authors violate anonymity rules in two ways:
   (1) they are are hiding authors / title of  a reference of own work  [25].
   There are clear instructions on the iswc-page that are clearly violated
   by this http://www.iswc.net/iswc15/calls/authorguide.html 
   (section ``Anonymous submission process'')
   (2) additionally they also give non-anonymized acknowledgements...
   
   
%%%%%%%%%%%%%%%%%%%%%%%%%%%%%%%%%%%%%%%
\section{Review 3}

Title: Dexterity Assesment for Salsa Dancers Through the Time-delay Embedded Phase Space Representation

\subsection{Confidence}
4  (Highly confident - I consider myself an expert in the area)

\subsection{Contribution to ISWC}
   The paper discusses the potential for the dexterity assessment for salsa
   dancers.  The authors present a sensing technique to distinguish the
   different level of the dexterity and show its feasibility through the
   sensor data collected from 3 salsa dancers.
   
\subsection{Overall Rating}
  3  (Maybe reject: I would agree with rejecting this paper.)
  
\subsection{The Review}
   The authors tackle an interesting topic, assessing the dexterity for
   salsa dancers. To this end, they develop a sensing technique using
   time-delay embedded series data. To show its feasibility, they collected
   the sensor data from 3 salsa dancers with different dexterity and
   described the phase space representation.

   One of my major concerns about the paper is its evaluation. First, the
   authors do not evaluate the end-to-end performance of the proposed
   sensing technique. With the collected data, they showed the intermediate
   results, cumulative energy percentage and 2-D reconstructed state spaces
   and their different pattern according to the dexterity, i.e., non-dancer,
   intermediate, and expert. However, the authors do not address the
   classification logic that classifies the final level of the dexterity,
   i.e., how the proposed solution finally recognizes the dexterity with the
   data. They do not evaluate the accuracy of the dexterity assessment.
   Second, the data set is too small to show the validity of the proposed
   technique. Since the authors used the sensor data from one participant
   per the dexterity, it is doubtful whether the data well represents the
   general dexterity of real dancers. Also, It was suspicious why the
   authors used only one non-dancer data although they collected the data
   from 11 non-dancers.

   Due to lack of detailed explanation and examples, it is difficult to
   understand how the proposed sensing technique operates. The authors
   present a number of terms, equations, and algorithms, but it is not easy
   to figure out how the algorithm works specifically, e.g., how the data is
   segmented, how the parameter values are determined, and how the dexterity
   is finally obtained.

   While the proposed idea is interesting, the need for the proposed
   technique is not well motivated throughout the paper. I think it would
   not be difficult for people to manually determine the dexterity of salsa
   dancers. What is the expected benefit of automatically detecting the
   dexterity? The paper would be stronger if the authors describe the
   motivating scenario of the proposed technique.

   
   
%%%%%%%%%%%%%%%%%%%%%%%%%%%%%%%%%%%%%%%
\section{Review 4}

\subsection{Confidence}
   3  (Very confident - I am knowledgeable in the area)
   
\subsection{Contribution to ISWC}
   This paper presents a useful application of wearable computing, which
   highlights its real-world implications. The paper, joined with other
   research work, demonstrates promising benefits that wearable devices can
   create for human.

\subsection{Overall Rating}
   3  (Maybe reject: I would agree with rejecting this paper.)
   
\subsection{The Review}
   This paper presents an approach to assess the dexterity of salsa dancers.
   Different from activity recognition, it aims to obtain more information
   to identify how well an action is performed. Clearly this research
   question is of interest to the community. The authors did solid work
   conducting the experiment. The presentation is clear.

   This paper has several weaknesses. First, it concentrates on a method
   called time-delay embedding. The motivation of using this approach is not
   strong. From the paper, we cannot see the necessity to apply this method.
   It would be better to compare the proposed approach with some existing
   methods and hopefully reveal performance advantages. The contribution is
   limited if the paper is about taking a new approach to an existing
   problem. The authors may want to provide elaborate more on why this
   method can outperform existing approaches and thus valuable to this
   problem.

   Dexterity of individual users can be complex. This paper takes a
   simplified model based on pattern matching. Additionally, sensory pattern
   of dexterity can be different on different performers due to different
   height, weight and so forth. In this case, it will be difficult to have a
   unified metric that would fit for different performers. It would be
   interesting to have several experienced dancers act like novices. We can
   see how the method and metrics work on individual performers. Also, more
   information about ground truth would help readers better assess the
   results.

   Overall, this paper presents interesting questions and a solid
   experiment. It still suffers from some weaknesses that worth refinements.


\end{document}
